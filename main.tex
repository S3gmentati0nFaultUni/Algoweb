\documentclass[a4paper]{book-enhanced}
\usepackage{src/preambolo}

\title{Nozioni per il corso di Algoritmica per il Web}
\subtitle{Dispense originali a cura di Sebastiano Vigna}
\author{Revisione di Alessandro Biagitotti}
\university{Università degli Studi di Milano}
\dept{Dipartimento di Informatica}
\logo{img/logo.tikz}
\subver{1.1.3}

\begin{document}

\maketitle
\noindent Le dispense di seguito presentate sono ad opera del professor Sebastiano Vigna (la
revisione è ad opera di Alessandro Biagiotti), si tratta di un collage di tutta una serie di
informazioni ricavate nell'arco del processo di preparazione all'esame, con il supporto di
Alessandro Clerici. Ringrazio anche Davide Polidori, il quale mi ha prestato i suoi appunti, senza i
quali alcuni passaggi sarebbero ancora oscuri.

L'intento delle dispense è quello di raccogliere le informazioni necessarie, fornendo una spiegazione per i passaggi meno chiari e aggiungendo parti inizialmente mancanti.

In questa nuova versione (1.1.3) ho cambiato la classe associata al progetto, passando da article a book-enhanced; il codice associato alla classe è stato leggerissimamente modificato da me e le modifiche originali sono ad opera di Alessandro Clerici.

\tableofcontents
\clearpage

\incl{NozioniDiBase}
\incl{Crawling}
\incl{Tecniche_di_distribuzione_del_carico}
\incl{Codici_istantanei}
\incl{Gestione_della_lista_dei_termini}
\incl{Risoluzione_delle_interrogazioni}
\incl{Centrality}
\incl{Information_retrieval}
\incl{Punteggi_endogeni}
\incl{Richiami_di_algebra_lineare}
\bibliography{Bibliography}
\bibliographystyle{alpha}
\incl{License}

\end{document}
